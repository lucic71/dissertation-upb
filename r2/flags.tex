\section{Compiler Configurations based on Undefined Behavior} \label{sec:flags}

After we defined the benchmark suite, we focued on evaluating the performance
impact of various undefined behavior optimizations. In this section we present
first the already-implemented configurations for the above mentioned
optimizations and second we present the configurations that we implemented.

Each configuration is made available through a compiler flag that can be used
when compiling the benchmarks.

\subsection{Already-implemented compiler flags}

We benefited from 5 flags in this category.

\textit{-fwrapv} instructs the compiler to assume that signed arithmetic
overflow of addition, subtraction and multiplication wraps around using
twos-complement representation. In LLVM, this has the impact of dropping the
\textit{nsw}~\cite{nsw-def} attribute in the above mentionted aritmetic
operations.

\textit{-fno-strict-aliasing} instructs the compiler to apply the strictest
aliasing rules available. In LLVM, this has the impact of dropping the
\textit{tbaa}~\cite{tbaa-def} attribute that is used for type based alias
analysis.

\textit{-fstrict-enums} instructs the compiler to optimize using the assumption
that a value of enumerated type can only be one of the values of the enumeration
(as defined in the C++ standard; basically, a value that can be represented in
the minimum number of bits needed to represent all the enumerators). In LLVM,
this has the impact of adding the \textit{range}~\cite{range-def} attribute to
memory operations.

\textit{-fno-delete-null-pointer-checks} instructs the compiler to assume that
programs can safely dereference NULL pointers and thus to not delete NULL
pointer checks that are proved to be redundant.

\textit{-fno-finite-loops} instructs the compiler to assume that no loop is
finite. In LLVM, this has the impact of dropping the
\textit{mustprogress}~\cite{mustprogress-def} attribute from all loops and
functions.

\subsection{Added compiler flags}

\textit{-fconstrain-shift-value} instructs the compiler to mask the
right-hand-side (RHS) of the shift operation so that it does not produce
undefined behavior when the RHS is bigger that the bitwidth. On x86, this add an
additional \textit{and} instruction for masking.

\textit{-fno-constrain-bool-value} instructs the compiler to not constrain bool
values to 0 and 1. In LLVM, this has the impact of dropping the \textit{range}
attribute from memory operations that work with booleans.

\textit{-fno-use-default-alignment} instructs the compiler to use alignment 1
for all memory operations including load, store, memcpy, etc. The alignments of
global variables and allocas remain unaffected. This has the impact of not
generating the most efficient code because the compiler cannot find the best
alignment for each operation that was forcefully aligned to 1.

\textit{-mllvm -zero-uninit-loads} instructs the compiler to replace
uninitialized loads with zero loads. This does not automatically initialize all
memory with zero, instead it fills the memory with zero only when the memory is
requested.

\textit{-fdrop-inbounds-from-gep -mllvm -trap-on-oob} instructs the compiler to
trap when it detects an out-of-bounds (OOB) memory access. By trapping, the
compiler is blocked from doing any further optimizations based on OOB. This is a
combination between a Clang flag (\textit{-fdrop-inbounds-from-gep}) and a LLVM
flag(\textit{-trap-on-oob}). We needed the Clang flag to make sure that no
optimization is triggered on the OOB access before we add the trap.

At the moment of writing this report, the last two flags are still in
development.

\subsection{Implementation details for added compiler flags}

\subsubsection{-fconstrain-shift-value}
To implement \textit{-fconstrain-shift-value} we had to do modifications in the
frontend of the comiler. The OpenCL specification already implemented this
behavior in EmitShl and EmitShr functions. Thus, we extented this by adding the
new flag for our purposes, as presented in
Listing~\ref{lst:fconstrain-shift-value}.

\begin{lstlisting}[style=Cstyle, caption={-fconstrain-shift-value
implementation}, label={lst:fconstrain-shift-value}]
@@ -3936,7 +3936,7 @@ Value *ScalarExprEmitter::EmitShl(const BinOpInfo &Ops) {
   bool SanitizeBase = SanitizeSignedBase || SanitizeUnsignedBase;
   bool SanitizeExponent = CGF.SanOpts.has(SanitizerKind::ShiftExponent);
   // OpenCL 6.3j: shift values are effectively % word size of LHS.
-  if (CGF.getLangOpts().OpenCL)
+  if (CGF.getLangOpts().OpenCL || CGF.CGM.getCodeGenOpts().ConstrainShiftValue)
     RHS = ConstrainShiftValue(Ops.LHS, RHS, "shl.mask");
   else if ((SanitizeBase || SanitizeExponent) &&
            isa<llvm::IntegerType>(Ops.LHS->getType())) {
@@ -4005,7 +4005,7 @@ Value *ScalarExprEmitter::EmitShr(const BinOpInfo &Ops) {
     RHS = Builder.CreateIntCast(RHS, Ops.LHS->getType(), false, "sh_prom");

   // OpenCL 6.3j: shift values are effectively % word size of LHS.
-  if (CGF.getLangOpts().OpenCL)
+  if (CGF.getLangOpts().OpenCL || CGF.CGM.getCodeGenOpts().ConstrainShiftValue)
     RHS = ConstrainShiftValue(Ops.LHS, RHS, "shr.mask");
   else if (CGF.SanOpts.has(SanitizerKind::ShiftExponent) &&
            isa<llvm::IntegerType>(Ops.LHS->getType())) {
\end{lstlisting}

The resulted LLVM IR for this change is presented in
Listing~\ref{lst:fconstrain-shift-value-result}. For the constrained version,
there is an additional and instruction introduced for clamping the RHS of the
shift operation.

\begin{lstlisting}[style=Cstyle, caption={-fconstrain-shift-value
results}, label={lst:fconstrain-shift-value-result}]
diff oversized-shift-constrain.ll oversized-shift-no-constrain.ll
<   %shl.mask = and i32 %1, 31
<   %shl = shl i32 %0, %shl.mask
---
>   %shl = shl i32 %0, %1
\end{lstlisting}

\subsubsection{-fno-constrain-bool-value}

For \textit{-fno-constrain-bool-value} we used the same strategy as above, i.e.
modify the frontend of the compiler to propagate the changes down to LLVM's
Intermediate Representation (IR), as presented in
Listing~\ref{lst:fnoconstrain-bool-value}.

\begin{lstlisting}[style=Cstyle, caption={-fnoconstrain-bool-value
implementation}, label={lst:fnoconstrain-bool-value}]
@@ -1683,7 +1683,7 @@ static bool getRangeForType(CodeGenFunction &CGF, QualType Ty,
 llvm::MDNode *CodeGenFunction::getRangeForLoadFromType(QualType Ty) {
   llvm::APInt Min, End;
   if (!getRangeForType(*this, Ty, Min, End, CGM.getCodeGenOpts().StrictEnums,
-                       hasBooleanRepresentation(Ty)))
+                       hasBooleanRepresentation(Ty) && CGM.getCodeGenOpts().ConstrainBoolValue))
     return nullptr;

   llvm::MDBuilder MDHelper(getLLVMContext());
\end{lstlisting}

The resulted LLVM IR for the above change will be similar to the one presented
in Listing~\ref{lst:fnoconstrain-bool-value-result}. For the constrained
(default) version, the !9 attribute is used for constraining the bool value
between 0 and 2. For the unconstrained version, the !9 attribute is dropped and
an additional and instruction is added in order to preserve the constraints of
the type. However because this lies in the domain of undefined behavior, the and
instruction is not required to be present.

\begin{lstlisting}[style=Cstyle, caption={-fnoconstrain-bool-value
results}, label={lst:fnoconstrain-bool-value-result}]
--- constrain-bool.ll
+++ no-constrain-bool.ll
 define dso_local noundef zeroext i1 @_Z3fooPb(ptr nocapture noundef readonly
%a) local_unnamed_addr #0 {
 entry:
-  %0 = load i8, ptr %a, align 1, !tbaa !5, !range !9
-  %tobool = icmp ne i8 %0, 0
+  %0 = load i8, ptr %a, align 1, !tbaa !5
+  %1 = and i8 %0, 1
+  %tobool = icmp ne i8 %1, 0
   ret i1 %tobool
 }

@@ -25,4 +26,3 @@
 !6 = !{!"bool", !7, i64 0}
 !7 = !{!"omnipotent char", !8, i64 0}
 !8 = !{!"Simple C++ TBAA"}
-!9 = !{i8 0, i8 2}
\end{lstlisting}

\subsubsection{-fno-use-default-alignment}

To implement this flag we had to modify many parts of Clang. Because we are
working with memory operations in this flag, we had to patch all calls to
CreateStore, CreateLoad, CreateMemCpy, etc~\cite{cgbuilder} methods. This was
a non-trivial task as there are tens of calls to these methods in different
subsystems of Clang.

After patching all usages, we discovered the problem of atomic loads and stores.
On x86, these types of memory operations need to be properly aligned, otherwise
the processor will trap. To solve this problem, we decided to link the compiled
applications with the atomic library provided by GCC which takes care of atomic
memory operations.

\subsubsection{-zero-uninit-loads and -trap-on-oob}

For implementing these flags we had do to modifications in the middle-end of the
copmiler. They are still in progress as the level of difficulty for implementing
them is greater than the previous flags.

For \textit{-zero-uninit-loads} the implementation difficulty is to find the
places where various types of variables get their initial values and replace
uninitialized values for each type with zero values for the corresponding types.

For \textit{-trap-on-oob} we plan to add traps for out-of-bounds accesses and
stop further optimizations from taking place. The difficulty for this task is to
spot all the places where the compiler exploits out-of-bounds accesses as there
are many places where this happens.
