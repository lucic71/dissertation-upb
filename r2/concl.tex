\section{Conclusions} \label{sec:ccl}

In this semester we started the work of evaluating the performance impact of
undefined behavior optimizations. We split the work in two parts. In the first
part we focused on developing a benchmark suite that contains C/C++ applications
with real loads, as opposed to synthetic loads. Then we started benchmarking 10
compiler flags that control the behavior of the compiler optimizations with
regards to undefined behavior. 5 of them were already implemented, but we also
added 5 new flags. Early results show that in nearly 90\% of the cases the
performance impact of undefined behavior in optimizations is insignificant.

\section{Further Work} \label{sec:further-work}

In the next semester we plan to explore new compiler configurations. This
includes optimizations based on use-after-free, optimizations based on alias
analysis that uses object-based rules or optimizations based on arithmetic
related undefined behaviors, such as division by 0.

We also plan to run the benchmarks on other hardware architectures such as AMD
or ARM to explore how they behave in comparison with the current hardware setup
that is based on Intel.

It's also an interest for us to combine the flags to see how they behave
together with regards to performance but must important is to analyze the
current impact and discover the root causes of current performance numbers.
