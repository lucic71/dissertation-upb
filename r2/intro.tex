\section{Introduction}

Our work is focused on evaluating the performance impact of undefined behavior
optimizations. Traditionally, undefined behavior was used as a free ticket to
issue various types of optimizations targeting either code speed or code size.
The impact of this type of optimization is already well known for
micro-benchmarks, however, there is no study covering the impact for real loads.
We fill this gap by benchmarking various instances of undefined behavior
optimizations in Clang/LLVM on real application loads written in C/C++.

The work in this semester was broken down into two parts. First, we started by
creating the infrastructure for running the benchmarks. Our main goal in this
regard was to gather a set of popular applications written in C and C++. After
exploring existing solutions for this problem, we decided to use the Phoronix
Test Suite for this task. The applications provided by Phoronix were later
compiled with various configurations that modify the behavior of the compiler
when exploiting undefined behavior. For each configuration, we gathered a set of
results that we will present later in this work.

Second, we were interested in exploring configurations for undefined behavior
exploitation. We started from a set of already-implemented configurations, such
as -fwrapv or -fno-strict-aliasing, and moved to configurations implemented by
us, such as -fno-constrain-bool-value. We benefited from 5 already-implemented
configurations and we managed to implement 5 more configurations. For some of
the configurations the level of difficulty was relatively low, because we only
had to modify the frontend of the compiler, i.e. Clang. However there existed
cases where we also had to modify the middle-end of the compiler, i.e. LLVM.

This work is structured as follows: Section~\ref{sec:benchmarks} provides
detailed information about the process of creating the benchmarks for our use
case, Section~\ref{sec:flags} talks about each compiler configuration that
modifies the behavior of the compiler when exploiting undefined behavior,
Section~\ref{sec:results} presents the results of the benchmarks for each
configuration, and Secton~\ref{sec:discussion} discusses the limitations of the
benchmarking process.
