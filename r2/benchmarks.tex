\section{Creating the Benchmarking Infrastructure} \label{sec:benchmarks}

To simplify the benchmarking process, we needed an infrastructure that would be
triggered for each undefined behavior configuration. The requirements for such
an infrastructure were: contain benchmarks for C/C++ applications and avoid
synthetic benchmarks. We wanted to avoid synthetic benchmarks because we aimed
to evaluate the impact of undefined behavior optimizations on real application
loads.

The investigation of the state-of-the-art benchmarking infrastructure for real
application loads resulted in two candidates, i.e. Phoronix Test
Suite~\cite{phoronix} and Geekbench~\cite{geekbench}. We chose to go further
with Phoronix because its software is open-source and it can be extended with
new benchmarks as per our needs, as opposed to Geekbench.

Phoronix offers a wide variety of benchmarks, written in various programming
languages. We were only interested in the ones written in C and C++. To do that
we had to filter all benchmarks by their dependencies and choose only the ones
that had a C/C++ compiler as a dependency. This resulted in a list of nearly 200
applications.

We had to further filter out the benchmark applications because a number of them
had problems while they were compiled with Clang. The effort of making them work
with this compiler was not worth as we had a good amount of applications that
already worked.

Next, we had the problem of benchmarking time. Spending too much time on
benchmarking was not beneficial for us so we decided to limit one round of
benchmarking to 24 hours. Because of that we had to further cut applications
that required a significant amount of time to benchmark. One example of such
application is GCC. One of the benchmarks in our suite was measuring the time of
building GCC form sources. GCC was a complicated 3-step build process, that
takes a few hours to complete, thus we wanted to avoid it.

This process resulted in the benchmark suite presented in Table~\ref{tab:suite}.
At this moment the suite mostly  contains applications that are either CPU bound
or memory bound. We discarded completly GPU, OpenMP and MPI applications as they
are more difficult to benchmark.
