\section{Introduction}

Our work is focused on evaluating the performance impact of undefined behavior
optimizations. Traditionally, undefined behavior was used as a free ticket to
issue various types of optimizations targeting either code speed or code size.
The impact of this type of optimization is already well known for
micro-benchmarks, however, there is no study covering the impact for real loads.
We fill this gap by benchmarking various instances of undefined behavior
optimizations in Clang/LLVM on real application loads written in C/C++.

In the past semester we compiled a benchmarking suite using Phoronix Test Suite
that focuses on various types of applications ranging from compression
algorithms to web servers. Besides that, we gathered and created various
configurations, controlled by compiler flags, that exploit undefined behavior in
the optimizations pipeline of LLVM.

In this semester we continued the work on developing new compiler flags for
controlling undefined behavior exploitation. Some of them were mentioned in the
Further Work section of the past report, i.e. flags for optimizations based on
use-after free, optimizations based on alias analysis that uses object-based
rules and optimization based on arithmetic related undefined behaviors, such as
division by 0. However, we also used Alive2~\cite{lopes2021alive2}, an automatic
verification tool for LLVM optimizations, to find new flags for our benchmarking
infrastructure. We found six more flags that are mainly concerned with various
LLVM attributes that can cause undefined behavior, for
example~\textit{noalias},~\textit{align} or~\textit{dereferenceable}. In total
we have 18 flags for controlling undefined behavior exploitation.

After finding all possible sources of undefined behavior, we carried on with
running the performance experiments. For this we compiled each benchmark in the
suite of 25 benchmarks with the following configurations:
\begin{itemize}
  \item baseline, i.e. without any undefined behavior flag
  \item one undefined behavior flag at a time
  \item all undefined behavior flags combined
\end{itemize}

This results in 20 builds for each benchmark that were run on the benchmarking
servers, i.e. an Intel Xeon E5-2680 v2 @ 3.60GHz and an arm Neoverse-N1 @ 3.00
GHz. To make sure that we get deterministic results for the benchmarks we issued
various techniques for stabilizing our builds, they include pinning the processes
to only one NUMA node and dropping the frequency of the processor.

After we gathered the first batch of results, we started analyzing them to
understand what parts of LLVM caused the performance regressions and how can we
recover the performance. Current results indicate losses of up to 15\% that
need careful investigation. At this moment we are investigating a performance
loss in simdjson caused by the absence of the~\textit{align} attribute on a
pointer argument from a performance critical function.
