\section{Finding new exploitations of undefined behavior}

We explored various flags that deal with the exploitation of undefined behavior.
Some of them were already integrated in LLVM, e.g. -fwrapv, and some of them were
developed by us based on our previous experience and based on
LangRef~\cite{langref}. However, there is a third category of flags that we
developed based on automatic exploration using Alive2.

Alive2~\cite{lopes2021alive2} is an automatic verification tool for LLVM
optimizations. Roughly, after each optimization pass it checks whether the
Intermediate Representation (IR) contains undefinedness. This helps us find
places in the optimization pipeline that exploit undefined behavior.

We decided to patch Clang, the frontend for C/C++, whenever we found an
undefined behavior exploitation in the optimization pipeline of LLVM. By making
sure that we generate code free of undefined behavior directly from the source,
i.e.  Clang, we made sure that further optimization cannot take any advantage or
further propagate the exploitation of undefined behavior. For example, instead
of patching each place in LLVM that takes advantage by division by 0, we
generated code from Clang that guarded each division so that LLVM cannot further
optimize the division.

% TODO: mention that we disabled inter procedural optimizations because Alive2 cannot
% work well with them.

The following is an example of undefined behavior detected using Alive2.
Listing~\ref{lst:is_sb_aq_enabled_c_code} contains a C function from the
aom-av1~\cite{aomav1benchmark} benchmark that returns a boolean value based on a
pointer received as parameter. Listing~\ref{lst:is_sb_aq_enabled_alive} contains
the corresponding Alive2 output for is_sb_aq_enabled. The output can be broken
down in multiple parts:

\begin{enumerate}
  \item The IR before the optimization and the IR before the optimization, they
are delimited by \(=>\)
  \item Information about the validity of the transformation, in this case we
get an "Transformation doesn't verify! (unsound) ERROR: Source has guardable UB"
  \item Detailed information about the cause of the error including values that
triggered the error and the memory state
\end{enumerate}

\begin{lstlisting}[style=Cstyle, caption={C code for is_sb_aq_enabled}, label={lst:is_sb_aq_enabled_c_code}]
static bool is_sb_aq_enabled(const AV1_COMP *const cpi) {
  return cpi->rc.sb64_target_rate >= 256;
}
\end{lstlisting}

\begin{lstlisting}[caption={Alive2 output on is_sb_aq_enabled}, label={lst:is_sb_aq_enabled_alive}]
1.
define i1 @is_sb_aq_enabled(ptr noundef %cpi) null_pointer_is_valid zeroext {
%entry:
  %cpi.addr = alloca i64 8, align 8
  store ptr noundef %cpi, ptr %cpi.addr, align 1
  %0 = load ptr, ptr %cpi.addr, align 1
  %rc = gep inbounds ptr %0, 677024 x i32 0, 1 x i64 427536
  %sb64_target_rate = gep inbounds ptr %rc, 248 x i32 0, 1 x i64 16
  %1 = load i32, ptr %sb64_target_rate, align 1
  %cmp = icmp sge i32 %1, 256
  ret i1 %cmp
}
=>
define i1 @is_sb_aq_enabled(ptr noundef %cpi) null_pointer_is_valid zeroext {
%entry:
  %rc = gep inbounds ptr noundef %cpi, 677024 x i32 0, 1 x i64 427536
  %sb64_target_rate = gep inbounds ptr %rc, 248 x i32 0, 1 x i64 16
  %0 = load i32, ptr %sb64_target_rate, align 1
  %cmp = icmp sge i32 %0, 256
  ret i1 %cmp
}

2.
Transformation doesn't verify! (unsound)
ERROR: Source has guardable UB

3.
Example:
ptr noundef %cpi = pointer(non-local, block_id=0, offset=-388623)

Source:
ptr %cpi.addr = null
ptr %0 = pointer(non-local, block_id=0, offset=-388623)
ptr %rc = poison
ptr %sb64_target_rate = poison
i32 %1 = UB triggered!

SOURCE MEMORY STATE
===================
NON-LOCAL BLOCKS:
Block 0 >       size: 601368    align: 9223372036854775808      alloc type: 0   address: 0
Block 1 >       size: 8968      align: 4        alloc type: 0   address: 601368

LOCAL BLOCKS:
Block 2 >       size: 601368    align: 9223372036854775808      alloc type: 0   address: 0

Target:
ptr %rc = poison
ptr %sb64_target_rate = poison
i32 %0 = UB triggered!
\end{lstlisting}

One can spot the poison values in \%rc and \%sb64_target_rate from the detailed
information section of the output. They cause the subsequent "UB triggered!"
because as per LangRef~\cite{load-semantics} when using a load instruction, "if
pointer is not a well-defined value, the behavior is undefined.". To stop this
exploitation of undefined behavior from taking place, we removed the
\textit{inbounds} keyboard from the GEP instruction directly from Clang. By
removing \textit{inbounds} we made sure that the GEP instructions where \%rc and
\%sb64_target_rate are created will not generate undefinedness as none of the
cases described in~\cite{getelementptr-semantics} is triggered, hence no
undefined behavior is triggered.

We integrated the functionality of removing \textit{inbounds} from GEP
instructions in a flag called -fdrop-inbounds-from-gep. It only takes effect in
Clang when different GEP instructions are generated. Using this method described
above, we created 5 more flags:

\begin{itemize}
  \item -fdrop-noalias-restrict-attr: Drop noalias attribute generated by the
restrict keyword
  \item -fdrop-align-attr: Drop align attribute
  \item -fdrop-deref-attr: Drop dereferenceable attribute
  \item -fdrop-ub-builtins: Drops builtin_assume_aligned and replaces builtin_unreachable with builtin_trap
  \item -fdrop-pure-const-attr: Drop pure and const function attributes
\end{itemize}

We also had plans to drop the noundef attribute but fortunately there already
existed a flag for that, callled "-Xclang -no-enable-noundef-analysis".

In the process of running Alive2 over the benchmarking suite, we found a couple
of bugs in LLVM and bugs in Alive2. Below are two examples of such bugs.

When running Alive2 over aom-av1 we found one bug in LLVM related to loop
vectorization~\cite{loopvectbug}, the LoopVectorizer was adding an misaligned
store because it was not respecting the alignment of the original store
instruction.

On BRL-CAD and OpenSSL we found a bug in Alive2 because it cannot analyse
locally-allocated pointers on function calls. Listing~\ref{lst:openssl_code} is
a snippet from OpenSSL that showcases the problem. Because abuf and bbuf are
pointers allocated inside xname_cmp, Alive2 cannot continue the analysis of the
pointers inside i2d_X509_NAME. This is not a fundamental problem in Alive2, the
functionality is not currently implemented.

\begin{lstlisting}[style=Cstyle, caption={OpenSSL code that was causing Alive2 bug}, label={lst:openssl_code}]

static int xname_cmp(const X509_NAME *a, const X509_NAME *b)
{
    unsigned char *abuf = NULL, *bbuf = NULL;

    ...

    // Alive2 cannot propagate the locally allocated pointers in i2d_X509_NAME
    alen = i2d_X509_NAME((X509_NAME *)a, &abuf);
    blen = i2d_X509_NAME((X509_NAME *)b, &bbuf);

    ...
}

\end{lstlisting}
