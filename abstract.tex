\begin{abstract} 
\noindent
The ISO C Standard added the undefined behavior notion as a mean to
portability. State-of-the-art compilers such as GCC and Clang/LLVM use
it to issue aggressive optimizations that break non-trivial pieces of
code used in real-life software. We argue that the performance impact
of undefined behavior (UB) optimizations in operating systems, such as
OpenBSD, is low.  Furthermore they introduce unobservable and
undocumented effects that have great impact of program robustness and
security. To test our hypothesis we take the compiler implementation
used in OpenBSD, i.e.  Clang/LLVM, and disable all undefined behavior
optimizations. Then we compare the performance of the system on multiple
hardware architectures with the above mentioned optimizations turned on
and off.
\end{abstract}

\begin{IEEEkeywords}
    compiler, optimization, undefined behavior, operating system
\end{IEEEkeywords}

