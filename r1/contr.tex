\section{Research Plan} \label{sec:rp}

In the second semester we plan to disable all undefined behavior
optimizations from the frontend of the compiler, i.e. Clang. Sources of
undefined behavior include:
\begin{itemize}
  \item signed integer overflow
  \item pointer overflow
  \item NULL pointer dereference
  \item shift overflow
  \item uninitialized load
  \item use-after-free
  \item out-of-bounds accesses
  \item infinite loop
\end{itemize}

In parallel, we plan to create a benchmarking infrastructure. We use
this infrastructure for detecting the performance changes for various
program categories. To achieve this, we compile each category with
different sets of undefined behavior optimizations. Doing this we
understand the impact of the frontend optimizations on each program
category.

In the third semester we tackle the optimizations present in LLVM and
the backend of the compiler. Compared to Clang, LLVM makes more
aggressive use of undefined behavior optimizations. In the first phase
we will disable the optimizations that we know about. Later, in the
second phase, we might use Alive2~\cite{lopes2021alive2} to find new
sources of undefined behavior that are used by LLVM.

For benchmarking we will use the same approach as in the first semester.
The plan is to make the modifications in LLVM easily configurable, i.e.
create flags that enable and disable the corresponding modifications.

We plan to continue this work in the fourth semester if we discover that
there is more modification work than we expected. Otherwise we will
focus on providing insights on the security side of undefined behavior
optimizations, as opposed to the performance side.
