\section{Related Work} \label{sec:rl}

Wang et al.~\cite{wang2012undefined} and Ertl~\cite{ertl2015every}
provide metrics for undefined based optimizations based on the SPECint
benchmark.

Wang et al. experimented with the consequences of disabling undefined
behavior optimizations on the SPECint 2006 benchmark. Out of 12 programs
in the benchmark they noticed slowdowns for 2 of them. 456.hmmer slows
down $7.2\%$ with GCC 4.7 and $9.0\%$ with Clang/LLVM 3.1 while
462.libquantum slows down $6.3\%$ with the same version of GCC and
$11.8\%$ with the same version of Clang/LLVM.

Ertl states that with Clang-3.1 and undefined behavior optimizations
turned on, the speedup factor is $1.017$ for SPECint 2006. Furthermore,
for a specific category of applications, i.e. Jon Bentley's traveling
salesman problem, the speedup factor can reach values greater than $2.7$
if the developer issues source-level optimizations by hand, surpassing
the undefined behavior optimizations issued by the compiler.

Clang/LLVM evolved in the methods of exploiting undefined behavior
optimizations and little research has been done in this area. We build
on top of the studies of Wang et al. and Ertl to provide a comprehensive
list of how compilers exploit undefined behavior to issue optimizations
and to provide the impact of this class of optimizations for various
application categories.
