\section{Conclusions} \label{sec:ccl}

The definition of undefined behavior is used by compiler
implementations to issue aggressive optimizations. We argue that this
class of optimization is very dangerous as it conflicts with programmer
intentionality and with a robust definition of code semantics. In this
study analyze the performance impact of undefined behavior optimizations
in real-life software projects. By doing so we evaluate if the
advantages of issuing UB optimizations surpass the security risks that
they introduce. In order to do this we take a robust and secure
implementation of operating system, i.e. OpenBSD, and compare the system
that contains UB optimizations with the same system without UB
optimizations. The comparison is done on multiple hardware architectures
to inspect what role UB optimizations have for various hardware setups.
