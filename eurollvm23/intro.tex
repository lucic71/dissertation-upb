\section{Context and Motivation}

The ISO C Standard~\cite{iso90} provides a definition of undefined
behavior that gives absolute freedom to compiler implementation when
erroneous program constructs, erroneous data or indeterminately-valued
objects are encountered. This allows Clang/LLVM to treat
undefined behavior in various ways while still being standard
conformant.

This allows the compiler implementation to use the definition of
undefined behavior to issue optimizations. However, the impact of such
optimizations is not clear. Some application categories might benefit
from the advantages of undefined behavior optimizations while other
might not.

In this context we take a set of application categories and assess the
performance impact of undefined behavior optimizations. The categories
are the following: webservers, circuit simulators, telephony, finance,
GUI, software defined radio, speech, compression, texture compression,
audio encoding, databases, chess, password cracking, cryptography,
security, parallel processing, image processing, bioinformatics,
simulation, video encoding, neural networks, HPC and compiler build
speed.

These categories are the result of a fine grained analysis of the
benchmarks provided by Phoronix Test Suite~\cite{pts}. We use this
benchmark framework as it provides support for a wide range of
applications and a fast and mature interface for running the benchmarks.

We test the appplication categories against different undefined behavior
optimization flags such as: -fwrapv, -fno-strict-aliasing,
-fstrict-enums, etc.
